% !TeX spellcheck = en_GB
\documentclass[]{subfiles}
\begin{document}
	\section{Properties of a convolution}
	 There are six properties for a convolution. These should be included on the formula sheet for the exam. However, you are expected to be able to explain what they mean and how they relate to each other.
	 \subsection{Linearity}
	 \begin{equation}
	 	f\ast (ag_1+bg_2)= a(f\ast g_1)  +b(f\ast g_2)
	 \end{equation}
 	\subsection{Commutativity}
 	\begin{equation}
 		f\ast g = g \ast f
 	\end{equation}
 \subsection{Associativity}
 \begin{equation}
 	f\ast (g\ast h) = (f\ast g) \ast h
 \end{equation}
\subsection{Causal functions}
\label{sec:causalFunctions}
\begin{equation}
	\left\{ \begin{array}{r@{\text{ = }}l}
		f(t)& 0\\
		g(t)&0
	\end{array}\right.
	\text{for } t<0
	\label{eq:StatementCausal}
\end{equation}
If the statement above (\ref{eq:StatementCausal}) is true, we can say these functions are causal. Furthermore we can state that:
\begin{align*}
	(f\ast g) (t) &= \int_{-\infty}^{\infty} f(\tau)g(t-\tau)d\tau\\
	&= \int_{0}^{t}f(\tau)g(t-\tau)d\tau
\end{align*}
\subsection{Finite interval}
\begin{equation}
	\left\{ \begin{array}{r@{\text{ for }}l}
		f(t)\neq 0& t\in\left[ 0,t_1\right] \\
		g(t)\neq 0&t\in \left[ 0,t_2\right] 
	\end{array}\right.
	\label{eq:StatementFiniteInterval}
\end{equation}
If the statement above (\ref{eq:StatementFiniteInterval}) is true, then $(f\ast g)(t)\neq 0$ for the interval $\left[0,t_1+t_2 \right] $.
\subsection{Delta function}
\begin{align*}
	x(t)\ast \delta(t-t_0) &= \int_{-\infty}^{\infty}\delta(\tau-t_0)x(t-\tau)d\tau\\
	&=  x(t-t_0)
\end{align*}
This will lead to a shift of the function to the right with a distance of $t_0$. 
\end{document}
