% !TeX spellcheck = nl
\documentclass[]{subfiles}
\begin{document}
	\section{Eigenschappen van een convolutie}
	 Er zijn zes eigenschappen die gelden voor een convolutie.
	 \subsection{Lineariteit}
	 \begin{equation}
	 	f\ast (ag_1+bg_2)= a(f\ast g_1)  +b(f\ast g_2)
	 \end{equation}
 	\subsection{Commutativiteit}
 	\begin{equation}
 		f\ast g = g \ast f
 	\end{equation}
 \subsection{Associativiteit}
 \begin{equation}
 	f\ast (g\ast h) = (f\ast g) \ast h
 \end{equation}
\subsection{Causale functies}
Als
\begin{equation}
	\left\{ \begin{array}{r@{\text{ = }}l}
		f(t)& 0\\
		g(t)&0
	\end{array}\right.
	\text{voor } t<0
\end{equation}
dan zijn deze functies causaal, en kunnen we zeggen dat:
\begin{align*}
	(f\ast g) (t) &= \int_{-\infty}^{\infty} f(\tau)g(t-\tau)d\tau\\
	&= \int_{0}^{t}f(\tau)g(t-\tau)d\tau
\end{align*}
\subsection{eindig interval}
Als
\begin{equation}
	\left\{ \begin{array}{r@{\text{ voor }}l}
		f(t)\neq 0& t\in\left[ 0,t_1\right] \\
		g(t)\neq 0&t\in \left[ 0,t_2\right] 
	\end{array}\right.
\end{equation}
dan zal $(f\ast g)(t)\neq 0$ voor het interval $\left[0,t_1+t_2 \right] $.
\subsection{Delta functie}
\begin{align*}
	x(t)\ast \delta(t-t_0) &= \int_{-\infty}^{\infty}\delta(\tau-t_0)x(t-\tau)d\tau\\
	&=  x(t-t_0)
\end{align*}
Dit zal leiden tot een een verschuiving van de functie naar rechts met een afstand $t_0$.
\end{document}
