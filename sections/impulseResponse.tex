% !TeX spellcheck = en_GB
\documentclass[]{subfiles}
\begin{document}
	\section{Impulse response $h(t)$}
	To analyse the response a system has when its input is an impulse, we first need to define some notation:
	\begin{equation*}
		y(t) = S:f(t)
	\end{equation*}
	This is how we denote the effects of  system $S$ when presented with input $f(t)$.   If we now apply the sampling property we get the following:
	\begin{equation*}
		y(t)=S:\int_{-\infty}^{+\infty}f(\tau)\delta(t-\tau)d\tau
	\end{equation*}
	Since S is an LTC system, we can move it inside of the integral:
	\begin{equation*}
		y(t)=\int_{-\infty}^{+\infty}f(\tau)\underbrace{\left[ S:\delta(t-\tau)\right] }_{h(t-\tau)}d\tau
	\end{equation*}
	This now becomes:
	\begin{equation*}
		y(t)=\int_{-\infty}^{+\infty}f(\tau)h(t-\tau)d\tau=f(t)\ast h(t)
	\end{equation*}
	This is a very important result: The zero  state  response  $y(t)$  of  a system  with  known  impulse  response $ h(t)$  is  the convolution of the excitation $x(t)$  with the impulse response. 
\end{document}