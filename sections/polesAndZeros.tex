% !TeX spellcheck = en_GB
\documentclass[]{subfiles}
\begin{document}
	\section{Poles and zeros of a system}
	In order to get more insight into the behaviour of an LTC system, we can take a closer look at the poles and zeros of the transfer function as defined in equation \ref{eq:def:transfer}. 
	\subsection{Zeros of a system}
	We can find the zeros of a system by solving the numerator equation for zero:
	\begin{equation}
		b_Ms^M+\ldots+b_1s^1+b_0=0 \Leftrightarrow S=z_i\in \mathbb{C}: i=1,\ldots,M
	\end{equation}
	Here, $z_i$ is a zero of the system
		\subsection{Poles of a system}
	We can find the Poles of a system by solving the denominator equation for zero:
	\begin{equation}
		s^N+s^{N-1}a_{N-1}+\ldots+s^1a_1+s^0a_0=0 \Leftrightarrow S=p_j\in \mathbb{C}: j=1,\ldots,N
	\end{equation}
	Here, $p_i$ is a pole of the system
	\subsection{Conclusion}
	We can make some observations based on the values of the poles:
	\begin{align}
		\forall j:& \text{Re}\{p_j\}<0 \Rightarrow \text{Stable}\\
		\exists j:& \text{Re}\{p_j\}>0 \Rightarrow \text{Unstable}\\
		\exists j:& \text{Re}\{p_j\}=0 \Rightarrow \text{Marginally stable}
	\end{align}
\end{document}